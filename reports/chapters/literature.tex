\chapter{Literature Review}
\label{ch:lit_rev} %Label of the chapter lit rev. The key ``ch:lit_rev'' can be used with command \ref{ch:lit_rev} to refer this Chapter.

The realm of social media, encompassing forums, social networking, microblogging, social 
bookmarking, and wikis \citep{esrc,gil2019}, significantly influences the dynamics of 
information dissemination. However, the unintentional factors contributing to the rise of fake 
news, as evidenced by incidents like the Nepal Earthquake case 
\citep{tandoc2017,radianti2016}, underline the intricacies of navigating the digital 
information landscape. In 2020, the global health sector encountered a substantial surge in 
fake news, prompting the World Health Organization (WHO) to declare an 'infodemic' during the 
COVID-19 outbreak. This infodemic involved a flood of both authentic and false information, 
including a noteworthy volume of misinformation.

In response to the challenges of identifying and combating fake news, several research 
initiatives have proposed innovative solutions. \cite{sahoo2021multiple} introduced an 
automatic fake news identification technique tailored for the Chrome environment, providing a 
means to detect fake news on Facebook. This approach leverages various features associated with 
a Facebook account, coupled with news content features, utilizing deep learning to analyze 
account characteristics.

FakeNewsNet, presented by \cite{shu2020fakenewsnet}, serves as a valuable repository of fake 
news data. This resource provides datasets with diverse features, spatiotemporal information, 
and social context, facilitating research in the domain of fake news. Evaluation indicates that 
user engagements can contribute to fake news detection in addition to news articles, 
highlighting the multifaceted nature of information dissemination.

\cite{Kumar} proposed a CNN and bidirectional LSTM ensembled network for identifying original 
and false news instances. Utilizing various advanced approaches, such as Long Short Term 
Memories LSTMs, Convolutional Neural Networks CNNs, attention mechanisms, and ensemble methods, 
the study collected news instances from sources like PolitiFact. The CNN and bidirectional LSTM 
ensembled network, incorporating an attention mechanism, demonstrated superior accuracy, 
emphasizing the significance of model complexity in addressing the fake news identification 
challenge.

Natural Language Processing (NLP) emerges as a pivotal tool in tackling fake news. 
\cite{choudhary2021linguistic} proposed a linguistic model, employing handcrafted linguistic 
features for fake news detection. The model, driven by language-specific features, demonstrated 
a remarkable 86\% accuracy in detecting and categorizing fake messages. Additionally, 
\cite{abdullah2020fake} adopted a multimodal approach, combining Convolutional Neural Network 
(CNN) and Long Short-Term Memory (LSTM), achieving significant performance in classifying fake 
news articles based on source, history, and linguistic cues.

Furthermore, \cite{aslam2021fake} introduced an ensemble-based deep learning model for 
classifying news as fake or real using the LIAR dataset. Employing a combination of Bi-LSTM-GRU-
dense and dense deep learning models, the study achieved notable accuracy, recall, precision, 
and F-score. Despite these advancements, ongoing research aims to enhance the robustness of 
these models, emphasizing the need for continual improvement and exploration of diverse 
datasets in fake news detection.



% A possible section of you chapter
\section{Evaluation of Existing Scholarship on Fake News Detection } % Use this section title or choose a betterone
The comprehensive examination of the literature reveals a multifaceted landscape in the realm 
of fake news identification. Researchers employ diverse strategies, ranging from linguistic 
models to multimodal approaches, emphasizing the need for a holistic understanding that 
incorporates both content and social context. Key findings underscore the significance of model 
complexity, with advanced architectures like the CNN bidirectional LSTM ensembled network 
exhibiting notable success. Natural Language Processing (NLP) emerges as a critical tool in 
deciphering news content, demonstrated by linguistic models and the utilization of NLP 
techniques for textual attribute analysis. Despite considerable progress, there remains a 
persistent call for improvement, urging researchers to explore feature richness, latent 
semantic features, and diverse datasets. The global impact of infodemics, particularly 
highlighted during the COVID-19 outbreak, underscores the urgency in developing robust fake 
news detection systems. In essence, the literature review illuminates the dynamic and evolving 
nature of fake news research, emphasizing innovation and adaptability in response to the 
challenges presented in the digital information age.
\\

% Pleae use this section
\section{Summary} 
In summary, this literature review provides a comprehensive exploration of the current state of 
research in the field of fake news detection. The chapter commences by delineating the 
landscape of social media and its role in the dissemination of misinformation. It delves into 
the inadvertent factors contributing to the emergence of fake news, exemplifying instances such 
as the Nepal Earthquake case. The review accentuates the gravity of the 'infodemic,' 
particularly evident during the COVID-19 outbreak, necessitating advanced detection mechanisms. 
Several notable research endeavors are scrutinized, including \cite{sahoo2021multiple} automatic fake 
news identification technique, \cite{shu2020fakenewsnet} repository, and \cite{Kumar}
CNN+bidirectional LSTM ensembled network. The significance of Natural Language Processing (NLP) 
in understanding and detecting fake news is highlighted through studies like \cite{choudhary2021linguistic} 
linguistic model. The literature underscores the need for continuous innovation, feature 
exploration, and adaptation to address the evolving challenges posed by the rampant spread of 
fake news. This synthesis of existing knowledge provides a robust foundation for the ensuing 
research endeavors aimed at enhancing the efficacy of fake news detection systems.~\\
