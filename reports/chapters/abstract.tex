%Two resources useful for abstract writing.
% Guidance of how to write an abstract/summary provided by Nature: https://cbs.umn.edu/sites/cbs.umn.edu/files/public/downloads/Annotated_Nature_abstract.pdf %https://writingcenter.gmu.edu/guides/writing-an-abstract
\chapter*{\center \Large  Abstract}
%%%%%%%%%%%%%%%%%%%%%%%%%%%%%%%%%%%%%%
% Replace all text with your text
%%%%%%%%%%%%%%%%%%%%%%%%%%%%%%%%%%%

The proliferation of misinformation, particularly in the form of fake news, has become a significant challenge in today's digital age. This research project aims to leverage Natural Language Processing (NLP) techniques and machine learning algorithms to detect and combat the spread of deceptive content in online news articles. By analyzing linguistic patterns and employing sentiment analysis, the study seeks to develop a robust framework for accurately identifying fake news articles. The research objectives include analyzing and preprocessing a corpus of news articles, developing a Long Short-Term Memory (LSTM) model for binary categorization, and evaluating the effectiveness of NLP techniques in detecting fake news sources. Through this investigation, the project aims to contribute to the development of reliable mechanisms for combating misinformation and promoting information integrity in online platforms. 


%%%%%%%%%%%%%%%%%%%%%%%%%%%%%%%%%%%%%%%%%%%%%%%%%%%%%%%%%%%%%%%%%%%%%%%%%s
~\\[1cm]
\noindent % Provide your key words
\textbf{Keywords:} fake news detection, natural language processing (NLP), machine learning, classifier comparison, text analysis 

\vfill
\noindent


