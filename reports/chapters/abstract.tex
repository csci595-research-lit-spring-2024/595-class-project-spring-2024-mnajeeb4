%Two resources useful for abstract writing.
% Guidance of how to write an abstract/summary provided by Nature: https://cbs.umn.edu/sites/cbs.umn.edu/files/public/downloads/Annotated_Nature_abstract.pdf %https://writingcenter.gmu.edu/guides/writing-an-abstract
\chapter*{\center \Large  Abstract}
%%%%%%%%%%%%%%%%%%%%%%%%%%%%%%%%%%%%%%
% Replace all text with your text
%%%%%%%%%%%%%%%%%%%%%%%%%%%%%%%%%%%

This study aims to develop a model using Natural Language Processing (NLP) techniques to differentiate between reliable and potentially unreliable news articles. Leveraging a dataset of news articles with labeled reliability indicators, we utilize attributes such as article titles, authors, and texts to extract linguistic features and sentiments. Our approach involves building various classifiers, including Logistic Regression, Stochastic Gradient Descent, Random Forest, GBC, XGBoost, Decision Tree, Multinomial Naive Bayes, and Bernoulli Naive Bayes. These classifiers are trained and evaluated based on the extracted features to determine their effectiveness in fake news detection. The study concludes with an analysis of the accuracy scores and confusion matrices to identify the most suitable classifier for this task. 


%%%%%%%%%%%%%%%%%%%%%%%%%%%%%%%%%%%%%%%%%%%%%%%%%%%%%%%%%%%%%%%%%%%%%%%%%s
~\\[1cm]
\noindent % Provide your key words
\textbf{Keywords:} fake news detection, natural language processing (NLP), machine learning, classifier comparison, text analysis 

\vfill
\noindent


