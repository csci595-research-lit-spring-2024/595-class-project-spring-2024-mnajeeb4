\chapter{Introduction}
\label{ch:into} % This how you label a chapter and the key (e.g., ch:into) will be used to refer this chapter ``Introduction'' later in the report. 
% the key ``ch:into'' can be used with command \ref{ch:intor} to refere this Chapter.



%%%%%%%%%%%%%%%%%%%%%%%%%%%%%%%%%%%%%%%%%%%%%%%%%%%%%%%%%%%%%%%%%%%%%%%%%%%%%%%%%%%
\section{Background}
\label{sec:into_back}
The pervasive dissemination of misinformation in online articles presents a critical challenge 
in today's information landscape. This project delves into the intersection of Natural Language 
Processing (NLP) and machine learning to address this issue. Motivated by the increasing impact 
of fake news on public perception and decision-making, the project seeks to contribute to the 
development of robust mechanisms for detecting and mitigating the spread of deceptive content. 

%%%%%%%%%%%%%%%%%%%%%%%%%%%%%%%%%%%%%%%%%%%%%%%%%%%%%%%%%%%%%%%%%%%%%%%%%%%%%%%%%%%
\section{Problem statement}
\label{sec:intro_prob_art}
Despite advancements in fake news detection techniques, accurately identifying deceptive content in online articles remains a significant challenge. Traditional machine learning approaches often struggle to capture the nuanced linguistic patterns and contextual information present in text data, leading to suboptimal performance in distinguishing between real and fake news articles. Therefore, there is a need to explore more sophisticated methods, such as LSTM networks, to improve the accuracy and effectiveness of fake news detection. The  dataset utilized in this research, referred to as the ISOT Fake News \cite{fake-news}

%%%%%%%%%%%%%%%%%%%%%%%%%%%%%%%%%%%%%%%%%%%%%%%%%%%%%%%%%%%%%%%%%%%%%%%%%%%%%%%%%%%
\section{Aims and objectives}
\label{sec:intro_aims_obj}
\textbf{Aims:} To enhance the precision of fake news detection within a corpus of news articles in \cite{fake-news} using LSTM networks. 

\textbf{Objectives:} 
\begin{enumerate}
    \item Evaluate the effectiveness of LSTM networks in discerning distinctive linguistic patterns associated with fake news sources within the defined news article corpus.\citep{fake-news} 
    \item Investigate the impact of LSTM-based models on improving the accuracy of fake news detection compared to traditional machine learning approaches. 
    \item Explore techniques for optimizing LSTM architectures, including hyperparameter tuning and model regularization, to achieve better performance in fake news classification tasks. 
\end{enumerate}
 



%%%%%%%%%%%%%%%%%%%%%%%%%%%%%%%%%%%%%%%%%%%%%%%%%%%%%%%%%%%%%%%%%%%%%%%%%%%%%%%%%%%
\section{Solution approach}
\label{sec:intro_sol} % label of Org section
The solution approach involves leveraging LSTM networks as the primary technique for fake news detection in online articles. The methodology includes the following steps:

\begin{enumerate}
    \item Preprocessing and tokenization of the fake news corpus to prepare the text data for LSTM modeling.
    \item Designing and training LSTM-based deep learning models for binary classification of news articles into real or fake categories.
    \item Fine-tuning the LSTM architectures and experimenting with different configurations to optimize model performance.
    \item Evaluating the trained LSTM models using standard evaluation metrics and comparing their performance against baseline models and traditional machine learning algorithms.
    \item Analyzing the results to assess the effectiveness of LSTM networks in addressing the fake news detection problem and identifying areas for further improvement.
\end{enumerate}
 

%%%%%%%%%%%%%%%%%%%%%%%%%%%%%%%%%%%%%%%%%%%%%%%%%%%%%%%%%%%%%%%%%%%%%%%%%%%%%%%%%%%
\section{Summary of contributions and achievements} %  use this section 
\label{sec:intro_sum_results} % label of summary of results



%%%%%%%%%%%%%%%%%%%%%%%%%%%%%%%%%%%%%%%%%%%%%%%%%%%%%%%%%%%%%%%%%%%%%%%%%%%%%%%%%%%
